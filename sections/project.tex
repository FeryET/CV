\documentclass[../professional-cv.tex]{subfiles}
\begin{document}
	\begin{rSection}{Projects}
		%--------------------------------------------------------------------------------
		%    Projects
		%-----------------------------------------------------------------------------------------------
		
		\begin{rProjectSection}
			{\href{https://github.com/FeryET}{Open Source Projects}}
			{ \em \fmtdate{5}{2020} - Present }
			{}{}			
			\item \href{https://github.com/FeryET/lda_classification}{\faGithub} Developed an LDA classification library using Python, Gensim, Tomotopy and XGBoost and demostrated an example in the accompanying \bluehref{https://medium.com/p/35d5b98d4f05}{blog post}.
			\item \href{https://github.com/FeryET/scratch_nn_lib}{\faGithub} Developed a neural network library from scratch supporting fully connected layers and sgd optimizer using Python, Numpy.
			\item \href{https://github.com/FeryET/persian-to-english-transformer}{\faGithub} Implemented and trained the famous Attention is All You Need transformer model for Persian to English translation task. Used huggingface tokenizers and dataset libraries and implemented the model from scratch in PyTorch.
			\item \href{https://github.com/FeryET/comic-cover-generator}{\faGithub} Implemented and trained StyleGanV2 on the task of generating comic covers collected from internet. Collected the data using selenium, and implemented the model in PyTorch via PyTorch-Lightning framework.
			\item \href{https://github.com/FeryET/pneumonia_analysis}{\faGithub} Implemented and trained MobileNetV2 on pneumonia classification task with the accompanying \bluehref{https://medium.com/@FeryET/a-cnn-to-classify-pneumonia-step-by-step-using-pytorch-13a90905abd7}{blog post}.
			
			
			
		\end{rProjectSection}
		\draft{
			\begin{rSubsection}
				{Domain Specific Papers Classifier, \em{\href{https://cogc.ir/}{COGC}}}
				{ \fmtdate{5}{2017} - \fmtdate{10}{2018} }
				{}{}
				\item Collected papers from google scholar, axiv and semanticscholar using the given keywords, then curated the collection by expert insights. Stack: Python, Selenium.
				\item Extracted abstracts and keywords from the curated collection using PDF2Text library on python and beautifulsoap on the paper page HTMLs.
				\item Vectorized and classified papers using a hybrid of Doc2Vec, Word2Vec + TFIDF weighting and SVM classifier. Stack: Gensim, Spacy, Scikit-Learn.
				\item Developed a QT gui using PyQT5.
				
			\end{rSubsection}	
		}
		
		
		%----------------------------------------------------------------------------------------
	\end{rSection}
\end{document}